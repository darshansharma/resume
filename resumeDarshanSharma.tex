\documentclass[line, margin, 12pt]{res}

\usepackage{hyperref}
\usepackage{amsmath,scalerel}
\usepackage{xcolor}
\hypersetup{
    colorlinks,
    linkcolor={red!50!black},
    citecolor={blue!50!black},
    urlcolor={blue!80!black}
}

\newcommand\aspace[1][5cm]{\hspace*{#1}} % To add horizontal space of 5cm
\DeclareMathOperator*{\Bigcdot}{\scalerel*{\cdot}{\bigodot}}


\begin{document}

\name{\aspace \huge\bf{\href{https://darshan.sh}{Darshan Sharma}}}
\address{\hspace{10mm}
~\textbullet~\href{mailto:thedarshansharma@gmail.com}{thedarshansharma@gmail.com} 
~\textbullet~+91-700-974-6321 
~\textbullet~{\href{https://darshan.sh}{darshan.sh}}~\aspace}

\begin{resume}

\section{INFO}
Full Stack Developer and Associate Cloud Engineer with \textbf{6} years of experience in designing and implementing scalable web applications

\section{EDUCATION}
B.E in Computer Science (Panjab University) - 2018

\section{EXPERIENCE}

\\ \textbf {Software Engineer at MonkAI \hspace{23mm} Dec 2019 - Jun 2024} \\
Improved React video player load time in the app feed by pre-loading just 25\% of the content, fetching the rest only after 90\% viewership of the initial segment, guided by user data analysis \\
TS: TypeScript, React, Node, AWS-Lambda, S3, MongoDB

Reduced DB expenses by 10\% by integrating Redis as a caching layer, resulting in a single definitive database call rather than 6-7 heavy mongodb calls. \\
TS: React, Nest, AWS-RDS, GraphQL, TypeScript \\
\\
\textbf{FullStack Developer at Paxcom \hspace{23mm} May 2019 - Nov 2019} \\
* Decreased the load on the node server backend as it was being used for heavy CPU computations using a load balancer and distributed equal load on 2 separate microservices.

* Achieved a 27\% decrease in server expenses by transitioning from a monolithic architecture to deploying numerous microservices, coupled with the integration of automated scaling and efficient traffic management.\\
\\
\\ \textbf{Software Engineer at Block8 \hspace{27mm} May 2018 - May 2019}
* Optimized a Node.js web application's performance by refining database queries, significantly improving runtime efficiency


\section{SKILLS}
\begin{itemize}
\item \textbf{Web:} HTML5, CSS3
\item \textbf{Languages:} JavaScript, TypeScript, Python
\item \textbf{Frameworks:} NodeJs, NestJs, ExpressJs, React
\item \textbf{Database:} MongoDB, PostgreSQL
\item \textbf{Tools:} GCP, AWS, Docker, CI/CD, Git
\end{itemize}

\section{PROJECTS}
\begin{itemize}
\item \textbf{Toyota:} Toyota Thailand website in React-Redux and Express with huge inventory management \\
Tech Stack: React, Typescript, Express, AWS, Jest, Git \\

\item \textbf{MyStake:} Share-trading application on geth, React, MongoDB, Nest and
enhanced React-Node app speed by 25\% by optimizing database queries and utilizing CDN for static assets. \\
Tech Stack: React, Typescript, Nest, AWS, Jest, Graphql, Docker \\

\item \textbf{Paxcel Website:} Reduced node server load by offloading heavy CPU computations (image processing, CSV to HTML conversion) using multithreading and microservices,enhancing server performance by 27\% and created microservices and deployment to aws \\
Tech Stack: React.Js, Typescript, Node, GCP, Kubernetes \\

\item \textbf{Block8 website:} At Block8 we were using React for the front end and Node for the back end. The database queries, which were written inside Lambda functions, were not optimized, performing the cross-operation first and then the selection operation. I tweaked it a bit to perform the selection operation first, followed by joins, and then the final selection. In this way, we were able to reduce some latency.\\
\item \textbf{MonkAI:} Improved React video player load time by 17\% using CloudFront for caching, resulting in better TTFB.
\end{itemize}


\end{resume}
\end{document}
