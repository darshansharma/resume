\documentclass[line, margin, 12pt]{res}

\usepackage{hyperref}
\usepackage{amsmath,scalerel}
\usepackage{xcolor}
\hypersetup{
    colorlinks,
    linkcolor={red!50!black},
    citecolor={blue!50!black},
    urlcolor={blue!80!black}
}

\newcommand\aspace[1][5cm]{\hspace*{#1}} % To add horizontal space of 5cm
\DeclareMathOperator*{\Bigcdot}{\scalerel*{\cdot}{\bigodot}}


\begin{document}

\name{\aspace \huge\bf{\href{https://www.darshansharma.me}{Darshan Sharma}}}
\address{
\href{mailto:thedarshansharma@gmail.com}{thedarshansharma@gmail.com} 
~\textbullet~+91-8437073522  ~\textbullet~\#227/2 \ Sector 46A\ Chandigarh, 160047\aspace
}


\begin{resume}
\section{AIM IN LIFE}
To use my skills in the best possible way for achieving the company's goals.

\section{EDUCATION}
B.E in Computer Science and Engineering \\
\hyperref[http://ccet.ac.in/]{Chandigarh College Of Engineering and Technology (Panjab University)} \\
CGPA: 8.11 / 10.0 \\ \\
\textbf{High School -} Kendriya Vidyalaya (CBSE) - 9.2 CGPA \\
\textbf{Secondary School -} Kendriya Vidyalaya (CBSE) - 84.2\%\\

\section{EXPERIENCE}
\begin{itemize}
\item \textbf{Software Engg. Intern at ZScaler \hspace{10mm} Feb, 2018-Jun, 2018}\\
Worked with SMCA team to create a Perl script that forms and use tunnels (GRE
and VPN) for an end to end communication. The script also performs authentication
(ADFS and SAML based) on different cloud nodes (ZEN). 
Created a web-portal in Django which shows the traffic flow of a cloud and options for user to start, stop, restart the traffic flow.
\\
\item \textbf{Software Engineer at Block8 \hspace{20mm} Sep, 2018-May, 2019}\\
Worked with myStake team. myStake is a platform which lets shareholder know their stake in a company. myStake uses ethereum (geth) blockchain to store data of its users. It also allows trading where buying/selling of shares happen. myStake uses Stripe as its payment gateway. It is built using meteorJS, reactJS(frontend), geth(blockchain). 
\\
\item \textbf{Software Engineer at Paxcom \hspace{25mm} May, 2019 - Present}\\
Working on kinator app. Tech stack is - Angular, NodeJS. Build features like manipulating data inside excel sheet (having formulas) in nodejs

\end{itemize}
\section{\hyperref[https://www.github.com/darshansharma]{PROJECTS}}
\begin{itemize}
\item \textbf{\href{https://github.com/darshansharma/CricHere/blob/master/src/test/Test.java}{CricHere}}\\
Fetched live score of ongoing cricket match using REST API in Java.
\item \textbf{\href{https://github.com/darshansharma/snorlax}{Snorlax}}\\
Terminal Application written in Python which gives you a short description of a Unix/Linux command.
\item \textbf{\href{https://web.archive.org/web/20160414063537/http://www.compuhelpindia.com:80/}{Web Development}}\\
Designed a full fledged website using HTML/CSS, PHP, MySQL for a tech institute in my city.
\item \textbf{Chat App(Console Based)}\\
Designed a chat system for two users to chat from terminal using socket programming. Transport layer protocol was TCP.
\item \textbf{Web Automation} \\
Used selenium web driver in python to execute a script that automates designs.
Creating a design on a T-Shirt is a manual task that takes a lot of time and energy of one but can be automated using selenium web driver API\\
\end{itemize}

\section{TECHNICAL SKILLS}
\begin{itemize}
\item \textbf{Area of Interest:} Data Structures, Algorithms, Neural Networks
\item \textbf{Programming Languages:} C, C++, Java, Python, JS, Bash
\item \textbf{Tools/Frameworks:} Django, Selenium WebDriver, MySQL, Git, WordPress, TensorFlow\\
\end{itemize}

\end{resume}
\end{document}
